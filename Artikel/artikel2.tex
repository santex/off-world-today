% Copyright 2009 Dominik Wagenfuehr <dominik.wagenfuehr@deesaster.org>
% Dieses Dokument unterliegt der Creative-Commons-Lizenz
% "Namensnennung-Weitergabe unter gleichen Bedingungen 3.0 Deutschland"
% [http://creativecommons.org/licenses/by-sa/3.0/de/].

\Abschnitt{THE TOWER}%
\Bild[absolut,boxwidth=0.7\paperwidth,boxalign=right]{helper}{moon anomaly}{The Zond3 space probe captures this image of a 27 mile high tower on the moon}

\begin{Artikel}{2009_10_gui}
{Lunar anomalies }
{Dominik Honnef/\\Dominik Wagenführ}
{The following Images are merely a starting point for researching older more "famous" Lunar Anomalies and this page will be developed over the next few weeks as the data is collected.
}

\Initial{1}
\fm{}
The Shard is an obvious structure which rises above the Moon's surface by more than a mile. Its overall irregular spindly shape (containing a regular geometric pattern) with constricted nodes and swollen internodes, if natural, has got to be a wonder of the Universe. No known natural process can explain such a structure. Computer enhancement with about 190 feet (60 meters) resolution shows an irregular outline with more reflective and less reflective surfaces. The amount of sunlight reflecting from parts of the Shard indicate a composition inconsistent with that of most natural substances. Only crystal facets and glass can reflect that much light (polished metallic surfaces are unnatural). Single crystals the size of city blocks are currently unknown. I concur with Hoagland that the Shard may be a highly eroded remnant of some sort of artificial structure made of glass-like material. Other larger structures and their reflectivity in the area support this theory.  NOTE: Need Source\fm{}
LO-III-84H
\fm{}
Lunar Orbiter LO-84M-III.


\end{Artikel}
